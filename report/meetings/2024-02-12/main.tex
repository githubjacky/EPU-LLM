% ! TeX program = xelatex
\documentclass[12pt]{beamer}
\usepackage{tabularx}
\usepackage{changepage}
\usepackage{float}
\usepackage{ragged2e}
\usepackage{amsmath}
\usepackage{graphicx}
\usepackage{appendix}
\usepackage[round]{natbib}   % omit 'round' option if you prefer square brackets
\bibliographystyle{aer}

\usepackage{xeCJK}
\setCJKmainfont{regular}[
    Path="../fonts/simplified_chinese/noto_serif/",
    BoldFont=bold.otf
]
\xeCJKDeclareSubCJKBlock{Hangul}{"1100 -> "11FF, "3130 -> "318F, "A960 -> "A97F, "AC00 -> "D7AF, "D7B0 -> "D7FF}
\setCJKmainfont{regular}[
    Hangul,
    Path="../fonts/korean/noto_serif/",
    BoldFont=bold.otf
]

\xeCJKDeclareSubCJKBlock{Kana}{"3040 -> "309F, "30A0 -> "30FF, "31F0 -> "31FF, "1B000 -> "1B0FF}
\setCJKmainfont{regular}[
    Kana,
    Path="../fonts/japanese/noto_serif/",
    BoldFont=bold.otf
]

\newcommand\setItemnumber[1]{\setcounter{enumi}{\numexpr#1-1\relax}}


\title{\huge{Auto-EPU}}
\author{Hsiu-Hsuan(Jacky) Yeh}
\date{2024-02-12}
\begin{document}
\maketitle
% \newpage% or \cleardoublepage
% % \pdfbookmark[<level>]{<title>}{<dest>}
% \pdfbookmark[section]{\contentsname}{toc}
% \tableofcontents
% \pagebreak


\begin{frame}{Add Defition Of EPU Into Prompt}
You are building indices of policy-related economic uncertainty based on
\{ country \} newspaper coverage frequency, with the aim to capture
uncertainty about who will make economic policy decisions, what economic policy
actions will be undertaken and when, and the economic effects of policy actions
(or inaction) – including uncertainties related to the economic ramifications
of “non-economic” policy matters, e.g., military actions.

The process of building the index is as follows: \newline
1. Define three sets of keywords, E, P, U, containing keywords
    corresponding to the economy, policy, and uncertainty, respectively. \newline
2. Given a collection of news articles x, an article is considered related to
    policy-related economic uncertainty if it "meets the following three
    criteria simultaneously":
\end{frame}


\begin{frame}{Add Defition Of EPU Into Prompt}
- Contains a word belonging to the E set \newline
- Contains a word belonging to the P set \newline
- Contains a word belonging to the U set \newline
3. The index is calculated as the number of news articles related to
    policy-related economic uncertainty divided by the total number of news
    articles in x. \newline

Your task is to "define and list \{n\_words\} keywords in bullet points for each E, P, U set". Eeach
keyword must be a "simple and general word", and thus exclude the composite
word, especially various uncertainty and policy categories.\newline

Response in \{ language \}  and no need of English translation.
\end{frame}


\begin{frame}{Compared to Human Annotation}{Policy Category}
\begin{table}[H]
\renewcommand\arraystretch{1.4}
% \caption{}
% \label{Taiwan policy keywords}
\begin{adjustwidth}{-0.0cm}{}
\begin{center}
\setlength{\tabcolsep}{25pt}
{
    \fontsize{14}{14} \selectfont
    {
\def\sym#1{\ifmmode^{#1}\else\(^{#1}\)\fi}
\begin{tabular}{l c c}
\hline\hline

\multicolumn{1}{c}{(1)}
&\multicolumn{1}{c}{(2)}
&\multicolumn{1}{c}{(3)}
\\

\multicolumn{1}{c}{Economy}
&\multicolumn{1}{c}{Policy}
&\multicolumn{1}{c}{Uncertainty}
\\
\hline
經濟              & 政策             & 不確定                  \\
成長               & 法規             & 風險                    \\
發展              & 立法             & 擔憂                    \\
投資               & 財政             & 不穩定                  \\
消費              & 金融             & 變數                    \\
生產              & 稅務             &                         \\
貿易              & 預算             &                         \\
\hline\hline
\end{tabular}
}

}
\end{center}
\end{adjustwidth}
\begin{justify}
{\tiny
    Note: this table show the converage percentage of LLM suggestion over
    human experts. The referenced keywoeds of Taiwan(TW) come from
    \cite{Chen2024}, China(CN) come from \cite{Huang2020}, Japan(JP) come from
    \cite{ArbatliSaxegaard2022}, and US, Australia(AU), South Korea(KR) come
    from \cite{Baker2016}
}
\end{justify}
\end{table}
\end{frame}


\begin{frame}{Add Defition Of EPU Into Prompt With Self Consistency}
You are building indices of policy-related economic uncertainty based on
\{ country \} newspaper coverage frequency, with the aim to capture
uncertainty about who will make economic policy decisions, what economic policy
actions will be undertaken and when, and the economic effects of policy actions
(or inaction) – including uncertainties related to the economic ramifications
of “non-economic” policy matters, e.g., military actions.

The process of building the index is as follows: \newline
1. Define three sets of keywords, E, P, U, containing keywords
    corresponding to the economy, policy, and uncertainty, respectively. \newline
2. Given a collection of news articles x, an article is considered related to
    policy-related economic uncertainty if it "meets the following three
    criteria simultaneously":
\end{frame}


\begin{frame}{Add Defition Of EPU Into Prompt with Self Consistency}
- Contains a word belonging to the E set \newline
- Contains a word belonging to the P set \newline
- Contains a word belonging to the U set \newline
3. The index is calculated as the number of news articles related to
    policy-related economic uncertainty divided by the total number of news
    articles in x. \newline

Please proceed with the following tasks step by step. \newline

1. "Define and list \{ n\_words \} keywords in bullet points for each E, P, U
    set". Each keyword must be a "simple and general word", and thus exclude
    the composite word, especially various uncertainty and policy categories.
\end{frame}


\begin{frame}{Add Defition Of EPU Into Prompt with Self Consistency}
2. Provide an example of a news article, including its title and content,
    which is related to policy-related economic uncertainty. \newline
3. Identify a word from the article in task 2 that belongs to the set
    defined in task 1 as E \newline
4. Identify a word from the article in task 2 that belongs to the set
    defined in task 1 as P \newline
5. Identify a word from the article in task 2 that belongs to the set
    defined in task 1 as U \newline

"Warning: The words listed in tasks 3, 4, and 5 must be consistent with those
you defined in task 1; do not list words that you did not define in the
E, P, U sets." \newline

Response in \{ language \}  and no need of English translation.
\end{frame}


\begin{frame}{Compared to Human Annotation}{Policy Category}
\begin{table}[H]
\renewcommand\arraystretch{1.4}
% \caption{}
% \label{Taiwan policy keywords}
\begin{adjustwidth}{-0.0cm}{}
\begin{center}
\setlength{\tabcolsep}{10pt}
{
    \fontsize{12}{12} \selectfont
    {
\def\sym#1{\ifmmode^{#1}\else\(^{#1}\)\fi}
\begin{tabular}{l c c c}
\hline\hline

\multicolumn{1}{c}{(1)}
&\multicolumn{1}{c}{(2)}
&\multicolumn{1}{c}{(3)}
&\multicolumn{1}{c}{(4)}
\\

\multicolumn{1}{c}{country}
&\multicolumn{1}{c}{vote}
&\multicolumn{1}{c}{definition}
&\multicolumn{1}{c}{definition with SC}
\\
\hline
TW & 8.571429 & 8.571429 & 17.142857 \\
CN & 12.903226 & 9.677419 & 16.129032 \\
JP & 6.250000 & 9.375000 & 12.500000 \\
KR & 17.391304 & 13.043478 & 13.043478 \\
US & 25.000000 & 37.500000 & 50.000000 \\
AU & 23.076923 & 15.384615 & 23.076923 \\
\hline\hline
\end{tabular}
}

}
\end{center}
\end{adjustwidth}
{\tiny
    Note: this table show the converage percentage of LLM suggestion over
    human experts. The referenced keywoeds of Taiwan(TW) come from
    \cite{Chen2024}, China(CN) come from \cite{Huang2020}, Japan(JP) come from
    \cite{ArbatliSaxegaard2022}, and US, Australia(AU), South Korea(KR) come
    from \cite{Baker2016}
    \cite{}
}
\end{table}
\end{frame}

\begin{frame}{References}
\bibliography{/Users/jackyyeh/Library/texmf/bibtex/bib/Zotero.bib}
\end{frame}
\end{document}
