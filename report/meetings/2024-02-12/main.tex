% ! TeX program = xelatex
\documentclass[11.5pt]{beamer}
\usetheme{metropolis}
\usepackage{tabularx}
\usepackage{changepage}
\usepackage{float}
\usepackage{ragged2e}
\usepackage{amsmath}
\usepackage{graphicx}
\usepackage{appendix}

\input{lib/fonts}


\title{\huge{Auto-EPU}}
\author{Hsiu-Hsuan(Jacky) Yeh}
\date{2024-02-12}
\begin{document}
\maketitle


\begin{frame}{Outline}
\tableofcontents
\end{frame}


\section{Add Defition Of EPU Into Prompt}


\begin{frame}{Add Definition}
You are building indices of policy-related economic uncertainty based on
\{ country \} newspaper coverage frequency, with the aim to capture
uncertainty about who will make economic policy decisions, what economic policy
actions will be undertaken and when, and the economic effects of policy actions
(or inaction) – including uncertainties related to the economic ramifications
of “non-economic” policy matters, e.g., military actions. \newline

The process of building the index is as follows: \newline
1. Define three sets of keywords, E, P, U, containing keywords
    corresponding to the economy, policy, and uncertainty, respectively.
\end{frame}


\begin{frame}{Add Definition}
2. Given a collection of news articles x, an article is considered related to
    policy-related economic uncertainty if it "meets the following three
    criteria simultaneously": \newline
- Contains a word belonging to the E set \newline
- Contains a word belonging to the P set \newline
- Contains a word belonging to the U set \newline

3. The index is calculated as the number of news articles related to
    policy-related economic uncertainty divided by the total number of news
    articles in x. \newline
\end{frame}


\begin{frame}{Add Definition}
"You must consider the concept of Word Segmentation and output non-compound
words, but rather simple and common words." \newline

Your task is to "define and list \{n\_items\} keywords in bullet points for
each E, P, U set". \newline

Output Instructions: \newline
1. List in bullet points \newline
2. \{ language\_instrucrions \}
\end{frame}




\section{Add Defition Of EPU Into Prompt With Self Consistency}


\begin{frame}{Add Defition with SC}
You are building indices of policy-related economic uncertainty based on
\{ country \} newspaper coverage frequency, with the aim to capture
uncertainty about who will make economic policy decisions, what economic policy
actions will be undertaken and when, and the economic effects of policy actions
(or inaction) – including uncertainties related to the economic ramifications
of “non-economic” policy matters, e.g., military actions. \newline

The process of building the index is as follows: \newline
1. Define three sets of keywords, E, P, U, containing keywords
    corresponding to the economy, policy, and uncertainty, respectively.
\end{frame}


\begin{frame}
2. Given a collection of news articles x, an article is considered related to
    policy-related economic uncertainty if it "meets the following three
    criteria simultaneously": \newline
- Contains a word belonging to the E set \newline
- Contains a word belonging to the P set \newline
- Contains a word belonging to the U set \newline

3. The index is calculated as the number of news articles related to \
policy-related economic uncertainty divided by the total number of news articles in x.
\end{frame}


\begin{frame}{Add Defition With SC}
"You must consider the concept of Word Segmentation and output non-compound
words, but rather simple and common words." \newline

Please proceed with the following tasks step by step. \newline

1. "Define and list \{ n\_items \} keywords in bullet points for each E, P, U
    set". \newline

2. Provide an example of a news article, including its title and content,
    which is related to policy-related economic uncertainty. \newline

3. Identify a word from the article in task 2 that belongs to the set
    defined in task 1 as E
\end{frame}


\begin{frame}{Add Defition with SC}

4. Identify a word from the article in task 2 that belongs to the set
    defined in task 1 as P \newline

5. Identify a word from the article in task 2 that belongs to the set
    defined in task 1 as U \newline

"Warning: The words listed in tasks 3, 4, and 5 must be consistent with those
you defined in task 1; do not list words that you did not define in the
E, P, U sets in task 1." \newline

Output Instructions: \newline
1. List in bullet points \newline
{language\_instructions}
\end{frame}


\section{Comparsion with Human Expert - Policy Category Across Countries}


\begin{frame}{Add Definition}
\begin{table}[H]
\renewcommand\arraystretch{1.2}
% \caption{}
% \label{Taiwan policy keywords}
\begin{adjustwidth}{-0.1cm}{}
\begin{center}
\setlength{\tabcolsep}{30pt}
{
    \fontsize{12}{12} \selectfont
    {
\def\sym#1{\ifmmode^{#1}\else\(^{#1}\)\fi}
\begin{tabular}{c c c c c}
\hline\hline

\multicolumn{1}{c}{(1)}
&\multicolumn{1}{c}{(2)}
&\multicolumn{1}{c}{(3)}
&\multicolumn{1}{c}{(4)}
&\multicolumn{1}{c}{(5)}
\\

\multicolumn{1}{c}{query 1}
&\multicolumn{1}{c}{quer 2}
&\multicolumn{1}{c}{quer 3}
&\multicolumn{1}{c}{quer 4}
&\multicolumn{1}{c}{quer 5}
\\
\hline
政府 & 政府 & 政府 & 政府 & 政府 \\
政策 & 政策 & 政策 & 政策 & 政策 \\
稅收 & 稅收 & 稅收 & 稅收 & 稅收 \\
社會福利 & 財政 & 預算 & 社會福利 & 社會福利 \\
教育 & 預算 & 財政 & 財政 & 健康保險 \\
健康保健 & 社會福利 & 社會福利 & 預算 & 教育 \\
環境保護 & 公共支出 & 福利改革 & 公共服務 & 勞工權益 \\
能源 & 社會保障 & 公共支出 & 貸款 & 環境保護 \\
醫療 & 法規 & 國家發展 & 能源 & 能源政策 \\
社會安全 & 立法 & 政黨 & 基礎建設 & 國防 \\
交通 & 政黨 & 政治 & 環境保護 & 財政 \\
公共服務 & 政治人物 & 行政 & 教育 & 公共交通 \\
勞工 & 選舉 & 立法 & 勞工 & 移民政策 \\
土地利用 & 政治改革 & 政府部門 & 醫療保健 & 土地利用 \\
社會服務 & 政治風險 & 政治改革 & 退休金 & 社會安全網 \\
基礎建設 & 公共服務 & 政治制度 & 社會安全 & 司法改革 \\
公共財政 & 政府部門 & 公民權利 & 都市計畫 & 治安 \\
文化 & 政策制定 & 法律 & 土地政策 & 資訊安全 \\
科技 & 政府決策 & 民主 & 市場監管 & 國際關係 \\
住房 & 政治氣候 & 政治體制 & 國家發展 & 文化政策 \\
\\
\hline\hline
\end{tabular}
}

}
\end{center}
\end{adjustwidth}
\begin{justify}
{\tiny
    Note: this table show the converage percentage of LLM suggestion over
    human experts. The referenced keywoeds of Taiwan(TW) come from
    \cite{Chen2024}, China(CN) come from \cite{Huang2020}, Japan(JP) come from
    \cite{ArbatliSaxegaard2022}, and US, Australia(AU), South Korea(KR) come
    from \cite{Baker2016}
}
\end{justify}
\end{table}
\end{frame}


\begin{frame}{Add Definition With SC}
\begin{table}[H]
\renewcommand\arraystretch{1.4}
% \caption{}
% \label{Taiwan policy keywords}
\begin{adjustwidth}{-0.0cm}{}
\begin{center}
\setlength{\tabcolsep}{10pt}
{
    \fontsize{12}{12} \selectfont
    {
\def\sym#1{\ifmmode^{#1}\else\(^{#1}\)\fi}
\begin{tabular}{l c c c}
\hline\hline

\multicolumn{1}{c}{(1)}
&\multicolumn{1}{c}{(2)}
&\multicolumn{1}{c}{(3)}
&\multicolumn{1}{c}{(4)}
\\

\multicolumn{1}{c}{country}
&\multicolumn{1}{c}{vote}
&\multicolumn{1}{c}{definition}
&\multicolumn{1}{c}{definition with SC}
\\
\hline
TW & 8.571429 & 8.571429 & 17.142857 \\
CN & 12.903226 & 9.677419 & 16.129032 \\
JP & 6.250000 & 9.375000 & 12.500000 \\
KR & 17.391304 & 13.043478 & 13.043478 \\
US & 25.000000 & 37.500000 & 50.000000 \\
AU & 23.076923 & 15.384615 & 23.076923 \\
\hline\hline
\end{tabular}
}

}
\end{center}
\end{adjustwidth}
{\tiny
    Note: this table show the converage percentage of LLM suggestion over
    human experts. The referenced keywoeds of Taiwan(TW) come from
    \cite{Chen2024}, China(CN) come from \cite{Huang2020}, Japan(JP) come from
    \cite{ArbatliSaxegaard2022}, and US, Australia(AU), South Korea(KR) come
    from \cite{Baker2016}
}
\end{table}
\end{frame}


\section{References}


\begin{frame}[allowframebreaks]{}
\renewcommand{\section}[2]{}%
\bibliography{/Users/jackyyeh/Library/texmf/bibtex/bib/Zotero.bib}
\end{frame}
\end{document}
